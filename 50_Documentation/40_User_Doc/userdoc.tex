% Options for packages loaded elsewhere
\PassOptionsToPackage{unicode}{hyperref}
\PassOptionsToPackage{hyphens}{url}
%
\documentclass[
]{article}
\usepackage[a4paper, top=0.75in, left=0.50cm, right=0.50cm]{geometry}
\usepackage{amsmath,amssymb}
\usepackage{lmodern}
\usepackage{iftex}
\ifPDFTeX
  \usepackage[T1]{fontenc}
  \usepackage[utf8]{inputenc}
  \usepackage{textcomp} % provide euro and other symbols
\else % if luatex or xetex
  \usepackage{unicode-math}
  \defaultfontfeatures{Scale=MatchLowercase}
  \defaultfontfeatures[\rmfamily]{Ligatures=TeX,Scale=1}
\fi
% Use upquote if available, for straight quotes in verbatim environments
\IfFileExists{upquote.sty}{\usepackage{upquote}}{}
\IfFileExists{microtype.sty}{% use microtype if available
  \usepackage[]{microtype}
  \UseMicrotypeSet[protrusion]{basicmath} % disable protrusion for tt fonts
}{}
\makeatletter
\@ifundefined{KOMAClassName}{% if non-KOMA class
  \IfFileExists{parskip.sty}{%
    \usepackage{parskip}
  }{% else
    \setlength{\parindent}{0pt}
    \setlength{\parskip}{6pt plus 2pt minus 1pt}}
}{% if KOMA class
  \KOMAoptions{parskip=half}}
\makeatother
\usepackage{xcolor}
\IfFileExists{xurl.sty}{\usepackage{xurl}}{} % add URL line breaks if available
\IfFileExists{bookmark.sty}{\usepackage{bookmark}}{\usepackage{hyperref}}
\hypersetup{
  hidelinks,
  pdfcreator={LaTeX via pandoc}}
\urlstyle{same} % disable monospaced font for URLs
\usepackage{longtable,booktabs,array}
\usepackage{calc} % for calculating minipage widths
% Correct order of tables after \paragraph or \subparagraph
\usepackage{etoolbox}
\makeatletter
\patchcmd\longtable{\par}{\if@noskipsec\mbox{}\fi\par}{}{}
\makeatother
% Allow footnotes in longtable head/foot
\IfFileExists{footnotehyper.sty}{\usepackage{footnotehyper}}{\usepackage{footnote}}
\makesavenoteenv{longtable}
\setlength{\emergencystretch}{3em} % prevent overfull lines
\providecommand{\tightlist}{%
  \setlength{\itemsep}{0pt}\setlength{\parskip}{0pt}}
\setcounter{secnumdepth}{-\maxdimen} % remove section numbering
\ifLuaTeX
  \usepackage{selnolig}  % disable illegal ligatures
\fi

\author{}
\date{}

\begin{document}
	
\setcounter{page}{28}

\hypertarget{user-documentation}{%
\section{A.4   User Documentation}\label{user-documentation}}

This document describes the installation and usage of the Rule
Extraction Assistant (REA).

\hypertarget{required-software}{%
\subsection{Required Software}\label{required-software}}

This project is based on Python and requires \texttt{python3.9}.

In order to install and run REA, the following python packages are
needed: - \texttt{pandas} - \texttt{numpy} - \texttt{scikit-learn} -
\texttt{tensorflow} - \texttt{rpy2}

They can be installed using \texttt{pip}, \texttt{venv},
\texttt{virtualenv}, etc. and the provided \texttt{Pifile} or
\texttt{requirements.txt}.

The following non-python software is needed for the rule extraction with
ALPA or DNNRE: - \texttt{R} (programming language/interpreter) - R
\texttt{C50} package

\hypertarget{example-setup}{%
\subsubsection{Example Setup}\label{example-setup}}

\begin{itemize}
\tightlist
\item
  Install python using your Linux distributions package manager or from
  the \href{https://www.python.org/}{official website}
\item
  Install R with the C50 library

  \begin{itemize}
  \tightlist
  \item
    intsall \texttt{R} using your Linux distributions package manager or
    download it from the
    \href{https://www.r-project.org/about.html}{official website}
  \item
    to install \texttt{C5.0}, invoke the \texttt{R} REPL on a command
    prompt (root privileges might be necessary to write to
    \texttt{usr/lib})
  \item
    type \texttt{install.packages("C50")}
  \item
    for this, you may need \texttt{build-essentials} (debian) or at
    least \texttt{gcc} and \texttt{gfortran}
  \end{itemize}
\item
  Set up the virtual environment

  \begin{itemize}
  \tightlist
  \item
    run \texttt{pipenv\ install\ -\/-python\ 3.9} in the source folder
  \end{itemize}
\end{itemize}

\hypertarget{usage}{%
\subsection{Usage}\label{usage}}

You can use the tool either through its API or through its CLI.

\hypertarget{api}{%
\subsubsection{API}\label{api}}

The api is documented in the \href{../20_API_Docs/html/index.html}{API
Documentation}, which is also contained in the \texttt{docs} folder of
the implementation.

\hypertarget{cli}{%
\subsubsection{CLI}\label{cli}}

We provide a CLI, which accepts a list of configuration files and
executes the specified pipeline run. Different runs can be achieved by
executing the CLI multiple times with different configuration file(s).
The generated output files of each module can then also be used by other
programs (provided that they can read the format). Some (advanced)
examples for the usage of the CLI can be found in the experiments folder
and the \texttt{run.sh} scripts in the hypothesis folders.

The next section provides you with an overview of all the configuration
parameters, their data-type, restrictions and also whether they are
required.

Execute the CLI by running \texttt{python\ -m\ rea\ -h} (in the source
folder or after installing). This will provide you with some help.

\hypertarget{configuration-format}{%
\subsection{Configuration Format}\label{configuration-format}}

\hypertarget{global-keys}{%
\subsubsection{Global Keys}\label{global-keys}}

\begin{longtable}[]{@{}
  >{\raggedleft\arraybackslash}p{(\columnwidth - 6\tabcolsep) * \real{0.2}}
  >{\centering\arraybackslash}p{(\columnwidth - 6\tabcolsep) * \real{0.4}}
  >{\centering\arraybackslash}p{(\columnwidth - 6\tabcolsep) * \real{0.10}}
  >{\centering\arraybackslash}p{(\columnwidth - 6\tabcolsep) * \real{0.3}}@{}}
\toprule
\begin{minipage}[b]{\linewidth}\raggedleft
Key name
\end{minipage} & \begin{minipage}[b]{\linewidth}\centering
Name in Code
\end{minipage} & \begin{minipage}[b]{\linewidth}\centering
optional/required
\end{minipage} & \begin{minipage}[b]{\linewidth}\centering
description
\end{minipage} \\
\midrule
\endhead
logging & GlobalKeys.LOGGING & optional & output verbosity. One of the
python logging modules log levels \\
seed & GlobalKeys.SEED & optional & seed used by all modules for
reproducibility \\
metrics\_filename & GlobalKeys.METRICS\_FILENAME & optional & filename
for the metrics generated by the extraction module \\
rules\_filename & GlobalKeys.RULES\_FILENAME & optional & filename for
the rules generated by the extraction module \\
predict\_instance\_filename & PREDICT\_INSTANCE\_FILENAME &
optional & name of the output pickle instance, is being used in
extraction and evaluation \\
\bottomrule
\end{longtable}

\hypertarget{keys-and-outputs-of-the-data-module}{%
\paragraph{Keys and Outputs of the
Data-Module}\label{keys-and-outputs-of-the-data-module}}

\begin{longtable}[]{@{}
  >{\raggedleft\arraybackslash}p{(\columnwidth - 6\tabcolsep) * \real{0.2}}
  >{\centering\arraybackslash}p{(\columnwidth - 6\tabcolsep) * \real{0.4}}
  >{\centering\arraybackslash}p{(\columnwidth - 6\tabcolsep) * \real{0.1}}
  >{\centering\arraybackslash}p{(\columnwidth - 6\tabcolsep) * \real{0.3}}@{}}
\toprule
\begin{minipage}[b]{\linewidth}\raggedleft
Key name
\end{minipage} & \begin{minipage}[b]{\linewidth}\centering
Name in Code
\end{minipage} & \begin{minipage}[b]{\linewidth}\centering
optional/required
\end{minipage} & \begin{minipage}[b]{\linewidth}\centering
description
\end{minipage} \\
\midrule
\endhead
input\_path & DataKeys.INPUT\_PATH & optional & path to the dataset from
which rules are to be extracted \\
output\_path & DataKeys.OUTPUT\_PATH & optional & Path to the folder to
fill with output. If directory doesn't exist, one will be created. \\
label\_col & DataKeys.LABEL\_COL & optional & index or name of the
column containing the labels, i.e.~the feature to be predicted \\
orig\_shape & DataKeys.ORIG\_SHAPE & optional & specifies the original
shape of the dataset before any conversion \\
test\_size & DataKeys.TEST\_SIZE & optional & percentage of the data
used for testing \\
dataset\_name & DataKeys.DATASET\_NAME & optional & friendly name to
give to the dataset \\
cat\_conv\_method & DataKeys.CAT\_CONV\_METHOD & optional & Method for
categorical conversion \\
categorical\_columns & DataKeys.CATEGORICAL\_COLUMNS & optional & List
of categorical columns to be converted in the dataset \\
scale\_data & DataKeys.SCALE\_DATA & optional & flag to MinMaxScale the
input data \\
\bottomrule
\end{longtable}

\begin{longtable}[]{@{}
  >{\raggedleft\arraybackslash}p{(\columnwidth - 2\tabcolsep) * \real{0.2}}
  >{\centering\arraybackslash}p{(\columnwidth - 2\tabcolsep) * \real{0.8}}@{}}
\toprule
\begin{minipage}[b]{\linewidth}\raggedleft
File
\end{minipage} & \begin{minipage}[b]{\linewidth}\raggedright
Description
\end{minipage} \\
\midrule
\endhead
\texttt{x\_train.npy} & Training data in numpy format \\
\texttt{y\_train.npy} & Trainin labels in numpy format \\
\texttt{x\_test.npy} & Test data in numpy format \\
\texttt{y\_test.npy} & Test labels in numpy format \\
\texttt{encoder.pickle} & Lable encoder, instance of the scikit
\texttt{LabelEncoder} serialized with pickle \\
\texttt{metadata.json} & Remaining important fields, like
\texttt{original\_size} \\
\bottomrule
\end{longtable}

\hypertarget{keys-and-output-for-the-model-module}{%
\subsubsection{Keys and Output for the
Model-Module}\label{keys-and-output-for-the-model-module}}

\begin{longtable}[]{@{}
  >{\raggedleft\arraybackslash}p{(\columnwidth - 6\tabcolsep) * \real{0.2}}
  >{\centering\arraybackslash}p{(\columnwidth - 6\tabcolsep) * \real{0.4}}
  >{\centering\arraybackslash}p{(\columnwidth - 6\tabcolsep) * \real{0.1}}
  >{\centering\arraybackslash}p{(\columnwidth - 6\tabcolsep) * \real{0.3}}@{}}
\toprule
\begin{minipage}[b]{\linewidth}\raggedleft
Key name
\end{minipage} & \begin{minipage}[b]{\linewidth}\centering
Name in Code
\end{minipage} & \begin{minipage}[b]{\linewidth}\centering
optional/required
\end{minipage} & \begin{minipage}[b]{\linewidth}\centering
description
\end{minipage} \\
\midrule
\endhead
nwtype & ModelKeys.TYPE & required & The type of network to use (``ff''
or ``conv'') \\
hidden\_layer\_units & ModelKeys.HIDDEN\_LAYERS & required & the number
layers and number of units for each hidden layer \\
hidden\_layer\_activations & ModelKeys.HIDDEN\_LAYER\_ACTIVATIONS &
required & specifies the activation function used for each hidden
layer \\
conv\_layer\_kernels & ModelKeys.CONV\_LAYER\_KERNELS & required &
kernel to be used in each layer of a convolutional network \\
use\_class\_weights & ModelKeys.USE\_CLASS\_WEIGHTS & optional & flag
that enables or disables precomputed class\_weights \\
batch\_size & ModelKeys.BATCH\_SIZE & optional & number of samples per
gradient update \\
epochs & ModelKeys.EPOCHS & optional & number of epochs to train the
model \\
output\_path & ModelKeys.OUTPUT\_PATH & required & path where to save
the model \\
learning\_rate & ModelKeys.LEARNING\_RATE & optional & Learning rate for
adam optimizer or initial learning rate for exponential decay \\
use\_decay & ModelKeys.USE\_DECAY & optional & flag that enables use of
adam with exponential decay \\
dropout & ModelKeys.DROPOUT & optional & Rate for keras \texttt{Dropout}
layer \\
val\_split & ModelKeys.VAL\_SPLIT & optional & percentage of the
training data used for validation \\
data\_path & ModelKeys.DATA\_PATH & required & path to the \texttt{Data}
output folder \\
\bottomrule
\end{longtable}

\begin{longtable}[]{@{}
  >{\raggedleft\arraybackslash}p{(\columnwidth - 2\tabcolsep) * \real{0.1977}}
  >{\raggedright\arraybackslash}p{(\columnwidth - 2\tabcolsep) * \real{0.8023}}@{}}
\toprule
\begin{minipage}[b]{\linewidth}\raggedleft
File
\end{minipage} & \begin{minipage}[b]{\linewidth}\raggedright
Description
\end{minipage} \\
\midrule
\endhead
\texttt{history.png} & Visualization of training-process measures
(validation loss/accuracy) \\
Keras Model Files & The trained tensorflow model \\
\bottomrule
\end{longtable}

\hypertarget{keys-and-output-for-the-extraction-module}{%
\subsubsection{Keys and Output for the
Extraction-Module}\label{keys-and-output-for-the-extraction-module}}

\begin{longtable}[]{@{}
  >{\raggedleft\arraybackslash}p{(\columnwidth - 6\tabcolsep) * \real{0.2}}
  >{\centering\arraybackslash}p{(\columnwidth - 6\tabcolsep) * \real{0.4}}
  >{\centering\arraybackslash}p{(\columnwidth - 6\tabcolsep) * \real{0.1}}
  >{\centering\arraybackslash}p{(\columnwidth - 6\tabcolsep) * \real{0.3}}@{}}
\toprule
\begin{minipage}[b]{\linewidth}\raggedleft
Key name
\end{minipage} & \begin{minipage}[b]{\linewidth}\centering
Name in Code
\end{minipage} & \begin{minipage}[b]{\linewidth}\centering
optional/required
\end{minipage} & \begin{minipage}[b]{\linewidth}\centering
description
\end{minipage} \\
\midrule
\endhead
trained\_model\_path & ExtractionKeys.MODEL\_PATH & required & path
where to load the model from \\
data\_path & ExtractionKeys.DATA\_PATH & required & path to the
\texttt{Data} output folder \\
algorithm & ExtractionKeys.ALGORITHM & required & extraction algorithm
to use, either ``alpa'' or ``dnnre'' \\
rules\_dir & ExtractionKeys.OUTPUT\_PATH & required & path of folder to
save rules and metrics \\
\bottomrule
\end{longtable}

\begin{longtable}[]{@{}
  >{\raggedleft\arraybackslash}p{(\columnwidth - 2\tabcolsep) * \real{0.2553}}
  >{\centering\arraybackslash}p{(\columnwidth - 2\tabcolsep) * \real{0.7447}}@{}}
\toprule
\begin{minipage}[b]{\linewidth}\raggedleft
File
\end{minipage} & \begin{minipage}[b]{\linewidth}\centering
Description
\end{minipage} \\
\midrule
\endhead
\texttt{eval\_metrics.json} & Contains metrics, such as execution time,
memory and best rho for ALPA \\
\texttt{rule\_classifier.pickle} & R C5.0 prediction instance serialized
with pickle \\
\texttt{rules.bin} & Extracted rules, serialized with pickle from the
internal format \\
\bottomrule
\end{longtable}

\hypertarget{keys-and-output-for-the-evaluation-module}{%
\subsubsection{Keys and Output for the
Evaluation-Module}\label{keys-and-output-for-the-evaluation-module}}

\begin{longtable}[]{@{}
  >{\raggedleft\arraybackslash}p{(\columnwidth - 6\tabcolsep) * \real{0.2}}
  >{\centering\arraybackslash}p{(\columnwidth - 6\tabcolsep) * \real{0.4}}
  >{\centering\arraybackslash}p{(\columnwidth - 6\tabcolsep) * \real{0.1}}
  >{\centering\arraybackslash}p{(\columnwidth - 6\tabcolsep) * \real{0.3}}@{}}
\toprule
\begin{minipage}[b]{\linewidth}\raggedleft
Key name
\end{minipage} & \begin{minipage}[b]{\linewidth}\centering
Name in Code
\end{minipage} & \begin{minipage}[b]{\linewidth}\centering
optional/required
\end{minipage} & \begin{minipage}[b]{\linewidth}\centering
description
\end{minipage} \\
\midrule
\endhead
trained\_model\_path & EvaluationKeys.MODEL\_PATH & required & path to
the trained tensorflow model \\
rules\_dir & EvaluationKeys.RULES\_DIR & required & path to the rules
folder \\
data\_path & EvaluationKeys.DATA\_PATH & required & path to the
\texttt{Data} output folder \\
evaluation\_dir & EvaluationKeys.OUTPUT\_PATH & required & path of
folder to save results of evaluation \\
\bottomrule
\end{longtable}

\begin{longtable}[]{@{}rc@{}}
\toprule
File & Description \\
\midrule
\endhead
\texttt{test\_eval.json} & Raw data produced by the evaluation on test
set \\
\texttt{test\_eval.md} & Evaluation report (based on raw data) for test
set \\
\texttt{train\_eval.json} & Raw data produced by the evaluation on train
set \\
\texttt{train\_eval.md} & Evaluation report (based on raw data) for
train set \\
confusion matrices & pngs of the confusion matrices \\
\bottomrule
\end{longtable}

\end{document}
